So, what have we really done and why does it make sense to use the GEV for such
problems? What kind of guarantees does the Fisher-Tipett-Gnedenko theorem give
us about the quality of the fit?


Well, the truth is, not too many.  

Remember that the block-maximum is defined as

\begin{equation}
  M_n \coloneqq \max\{ X_1, X_2, \dots, X_n \}
\end{equation}

First notice the following exact equality:


\begin{equation}
    P(M_n < z) = P(X_1< z \text{ and } X_2 < z ... \text{ and  } X_n < z) = F^n(z)
\end{equation}


So, if we know the cumulative distribution, there is no need to resort to the
GEV. Typically, of course, we do not know it. The above equality implies:


\begin{equation}
    \lim P(M_n < z) =     
        \begin{cases}
          0 & \text{if}\ F(z) < 1 \\
          1 & \text{otherwise}
        \end{cases}
\end{equation}


We actually always know the exact limit of the distribution of the block-maxima!
It is degenerate (either a step function of identical zero). In fact, this
degenerate distribution can be seen as a limit of the GEV. It would correspond
to normalizing constants $a_n=1, \ b_n=0$.

While this observation is very simple and the difference between the cdf of
block maxima $P(M_n < z)$ and its degenerate limit does decrease as $n$
increases, this limiting distribution is unexpressive and fitting it to data
does not provide probabilistic insight.

\hrulefill\\*

\textbf{Q:} How many parameters does the exact limit of $F^n$ have? What would we get if we fit it to data?

\textbf{A:}

\hrulefill\\*

Introducing the normalizing constants $a_n$ and $b_n$ \textit{might} allow the
distribution of renormalized block maxima to converge to something non-trivial.
It also might not.

In applications we usually care about modeling $M_n$ for a \textit{fixed $n_0$} (or
maybe for a few selected $n_i$). An arbitrary series of $a_n$ and $b_n$ that at
some point helps convergence does not directly address our needs. In fact, this
is also not what we do \textemdash\ by fitting the GEV parameters to data for our selected
$n_0$ we automatically find the \textit{best} $a_{n_0}$ and $b_{n_0}$ that minimize the
difference between $F^{n_0}(z)$ and $G(z)$.

Clearly $G(z)$ is much more expressive than the degenerate exact limit and could
potentially provide a good fit. So, the convergence that we really care about is
to answer the question:

How well do the best fits of $G(z)$ for fixed $n$ \textemdash\ let us call them
$G_n(z)$ \textemdash\ approximate the distributions $F^n(z)$ as $n$ increases?
One could e.g. be interested in the infinity norm

\begin{equation}
  \Delta_n := \sup_z | F^n(z) - G_n(z) |
\end{equation}


This is not the same as asking how well $G(z)$ approximates some rescaled
variant of $F^n(z)$ with $n$-dependent normalization constants! That would be


\begin{equation}
  \tilde{\Delta}_n(a_n, b_n) := \sup_z |F^n(a_n z + b_n) - G(z) |
\end{equation}


In the latter question, the choice of normalization constants matters, in the
former it does not \textemdash\ they are implicitly determined by the best fit for each
$n$. Since for $\Delta_n$ the $a_n, b_n$ have been optimized, one could
reasonably expect a relation of the type


\begin{equation}
  \Delta_n \approx \min_{a_n, b_n} \tilde{\Delta}_n(a_n, b_n)
\end{equation}


to hold.

It is easy to see that given some normalizing sequences $a_n, b_n$, the
convergence to a GEV is possible, than with other sequences $\tilde{a}_n,
\tilde{b}_n$ with some $a>0, b$ such that


\begin{equation}
  \lim_{n\rightarrow \infty} \frac{\tilde{a}_n}{a_n} = a \quad,\quad \lim_{n \rightarrow \infty} \frac{b_n-\tilde{b}_n}{a_n} = b
\end{equation}


the rescaled $\frac{M_n-\tilde{b}_n}{\tilde{a_n}}$ also converges to a GEV of
the same type (with the same $\xi$). This is often formulated that a
distribution $F$ has \textit{a fixed domain of attraction}. However, the error rates
$\tilde{\Delta}_n(\tilde{a}_n, \tilde{b}_n)$ would be different from those
associated to $a_n, b_n$.

Unfortunately, theoretical bounds for the quantity of interest $\Delta_n$ are
hard to come by \textemdash\ we are not aware of any. They also highly depend on the
fitting procedure, which is non-trivial, as we have seen above. There are some
bounds for quantities of the type $\tilde{\Delta}_n(\tilde{a}_n, \tilde{b}_n)$
(see the annotated literature reference) but they are rather loose and not
really helpful in practice. Therefore, the EVT theorems are more of a motivation
for selecting distribution families for fitting than a rigorous approach with
guarantees. In practice the convergence and fit tend to work pretty well,
though.
