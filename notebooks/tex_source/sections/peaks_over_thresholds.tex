So far we have only used the first theorem of EVT. As you might have noticed
above, it can be somehow wasteful when it comes to data efficiency. Since the
GEV is fitted on block-maxima, a huge number of data points remain unused for
parameter estimation. The second theorem of EVT gives rise to a more efficient
approach

The above equation can be used to estimate the entire tail of the cdf $F$ of $X$
from a sample of size $N$ obtained by sampling repeatedly from $F$. First note
that for a single $u$ we can approximate the cdf through the sample statistics
as:


\begin{equation}
1-F(u) = P(X>u) \approx \frac{N_u}{N}
\end{equation}


where $N_u$ is the number of samples with values above $u$. Interpreting $u$ as
a threshold, we will call those samples \textit{peaks over threshold} (PoT) and
$N_u$ is simply their count.

\hrulefill\\*

\textbf{Q:} What should $u$ and the data set fulfill in order for the above
approximation to be accurate?

\textbf{A:} It should be small enough such that many data points are larger than it.
Then the approximation in $P(X>u) \approx \frac{N_u}{N}$ holds (the estimator is
not too biased). 

\hrulefill\\*

Now we can perform a series of approximations for $z>u$ to get to the
tail-distribution. First using $P(X>u) \approx \frac{N_u}{N}$ we get


\begin{align}
    \begin{split}
        P(X>z) &= P(X>z \cap X>u)  \\
        &= P(X>z \mid X>u) P(X>u)  \\
        &\approx \frac{N_u}{N} P(X>z \mid X>u).   
    \end{split}
\end{align}


Now we use the GDP theorem to approximate


\begin{align}
    \begin{split}
        P(X>z \mid X>u) &= P(X-u > z -u \mid X>u)  \\ 
        &\approx \left( 1 + \frac{\xi (z-u)}{\tilde{\sigma}} \right)^{-\frac{1}{\xi}}.
    \end{split}
\end{align}


Putting everything together gives


\begin{equation}
    P(X>z) \approx \frac{N_u}{N}  \left( 1 + \frac{\xi (z-u)}{\tilde{\sigma}} \right)^{-\frac{1}{\xi}}.
\end{equation}


\hrulefill\\*

\textbf{Q:} Intuitively, what does $u$ need to fulfill for both approximations to
hold?

\textbf{A:} $u$ should be small enough such that the approximation $P(X>u) \approx
\frac{N_u}{N}$ holds and sufficiently large such that the generalized pareto
distribution is a good estimate of the tail of the distribution for values
larger than $u$. Intuitively, it should be at the *beginning of the tail*, where
for values larger than $u$ only the tail behavior plays a role - i.e. no more
local extrema or other specifics of the underlying distribution of the data.

\hrulefill\\*
